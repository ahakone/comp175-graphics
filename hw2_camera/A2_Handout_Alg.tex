\documentclass[10pt,twocolumn]{article}
\usepackage[margin=0.75in]{geometry}                
% See geometry.pdf to learn the layout options. There are lots.
\geometry{letterpaper}                   % ... or a4paper or a5paper or ... 
%\geometry{landscape}                % Activate for for rotated page geometry
%\usepackage[parfill]{parskip}    % Activate to begin paragraphs with an empty line rather than an indent
\setlength{\columnsep}{1cm}
\usepackage{graphicx}
\usepackage{amssymb}
\usepackage{epstopdf}
\usepackage{fullpage}
\usepackage[usenames]{color}
\usepackage{titlesec}
\usepackage{hyperref}
\usepackage{framed}
\usepackage{mathtools}

\definecolor{light-gray}{gray}{0.45}
\titleformat{\section}
{\color{black}\normalfont\Large\bfseries}
{\color{black}\thesection}{1em}{}

\titleformat{\subsection}
{\color{light-gray}\normalfont\large\bfseries}
{\color{light-gray}\thesubsection}{1em}{}

\DeclareGraphicsRule{.tif}{png}{.png}{`convert #1 `dirname #1`/`basename #1 .tif`.png}

\title{\Huge{\bf Assignment 2: Camera}}
\author{Comp175: Introduction to Computer Graphics -- Spring 2016}
\date{Algorithm due:  {\bf Wednesday Feb 24th} at 11:59pm\\
Project due:  {\bf Wednesday March 2nd} at 11:59pm}                            
               % Activate to display a given date or no date

\begin{document}
\maketitle
%\section{}
%\subsection{}

\begin{verbatim}
Your Names: ______Anzu Hakone____________

            ______Megan Van Welie________

Your CS Logins: ____ahakon01_____________

                ____mvanweli_____________
\end{verbatim}

\section{Instructions}
Complete this assignment only with your teammate. You may use a
calculator or computer algebra system. All your answers should be given in simplest form.
When a numerical answer is required, provide a reduced fraction (i.e. 1/3) or at least three
decimal places (i.e. 0.333). Show all work; write your answers on this sheet. This algorithm handout is worth 3\% of your final grade for the class.

\vspace{-0.4cm}
\section{Axis-Aligned Rotation}

Even though you won't have to implement Algebra.h, you will need to know (in principle) how to do these operations. For example, you should know how to create a rotation matrix...

\noindent{\tt Matrix rotX\_mat(const double radians)}\\
{\tt Matrix rotY\_mat(const double radians)}\\
{\tt Matrix rotZ\_mat(const double radians)}\\\\
Each of these functions returns the rotation matrix about the $x$, $y$, or $z$ axis (respectively) by the specified angle.
\begin{framed}
\noindent {\bf [1 point]} Using a sample data point (e.g., (1, 1, 1)), find out what happens to this point when it is rotated by 45 \emph{degrees} using each of these matrices.
\end{framed}

\begin{framed}
%\vspace{20em}
point = $\begin{bmatrix}
1 \\
1 \\
1 \\
1 \\
\end{bmatrix}$\\
rotX = $\begin{bmatrix}
1 \\
0 \\
\sqrt{2} \\
1 \\
\end{bmatrix}$
rotY = $\begin{bmatrix}
\sqrt{2}\\
1 \\
0 \\
1\\
\end{bmatrix}$
rotZ = $\begin{bmatrix}
0 \\
\sqrt{2}\\
1 \\
1\\
\end{bmatrix}$

\end{framed}

\section{Arbitrary Rotation}
Performing a rotation about an arbitrary axis can be seen as a composition of rotations around the bases. If you can rotate all of space around the bases until you align the arbitrary axis with the $x$ axis, then you can treat the rotation about the arbitrary axis as a rotation around $x$. Once you have transformed space with that rotation around $x$, then you need to rotate space back so that the arbitrary axis is in its original orientation. We can do this because rotation is a rigid-body transformation.\\\\
To rotate a point p, this series of rotations looks like:
\[p' = M^{-1}_1\cdot M^{-1}_2 \cdot M_3 \cdot M_2 \cdot M_1 \cdot p\]
where $M_1$ rotates the arbitrary axis about the $y$ axis, $M_2$ rotates the arbitrary axis to lie on the $x$ axis, and $M_3$ rotates the desired amount about the $x$ axis.

\subsection{Rotation about the origin}
\begin{framed}
\noindent{\bf [1.5 points]} Assuming we want to rotate $p=(p_x,p_y,p_z,1)$ about a vector $a = \langle a_x,a_y,a_z,0\rangle$ by $\lambda$ radians, how would you calculate $M_1$, $M_2$, and $M_3$ in terms of the functions {\tt rotX\_mat}, {\tt rotY\_mat}, and {\tt rotZ\_mat}?\\\\
\emph{Hint: This is tough! You can (and should) compute angles to use along the way. You can accomplish arbitrary rotation with {\tt arctan} and/or {\tt arcos}, for instance. Approach this problem step by step, it's not extremely math heavy. You should need only one {\tt sqrt} call.}
\end{framed}

\begin{framed}
$M_1$ = rotY\_mat($\arctan(\frac{z}{x})$)\\
$M_2$ = rotZ\_mat($\arcsin(\frac{y}{\sqrt{z^2 + x^2}})$)\\
$M_3$ = rotX\_mat($\lambda$)\\
\end{framed}

\subsection{Rotation about an arbitrary point}
\begin{framed}
\noindent{\bf [1 point]} The equation you just derived rotates a point $p$ about the origin. How can you make this operation rotate about an arbitrary point $h$?
\end{framed}

\begin{framed}
    p' = $T^{-1} \cdot M_1^{-1} \cdot M_2^{-1} \cdot M_3 \cdot M_2 \cdot M_1 \cdot T \cdot p$
\end{framed}


\section{Camera Transformation}
To transform a point $p$ from world space to screen space we use the \emph{normalizing transformation}. The normalizing transformation is composed of four matrices\footnote{Technically, there are five matrices. The last step is the 2D viewport transform, which is omitted here because OpenGL takes care of this step for us.}, as shown here:
\[p' = M_1 \cdot M_2 \cdot M_3 \cdot M_4 \cdot p\]
Each $M_i$ corresponds to a step described in lecture. Here, $p$ is a point in world space, and we would like to construct a point $p'$ relative to the camera's coordinate system, so that $p'$ is its resulting position on the screen (with its $z$ coordinate holding the depth buffer information). You can assume that the camera is positioned at $(x,y,z,1)$, it has look vector $look$ and up vector $up$, height angle $\theta_h$, width angle $\theta_w$, near plane $near$ and far plane $far$.
\begin{framed}
\noindent {\bf [1/2 pt. each]} Briefly write out what each matrix is responsible for doing. Then write what values they have. Make sure to get the order correct (that is, matrix $M_4$ corresponds to only one of the steps described in lecture.)\\

$M_1:$ Unhinge near clipping frame
$\begin{bmatrix}
    1 & 0 & 0 & 0 \\
    0 & 1 & 0 & 0 \\
    0 & 0 & \frac{-1}{c+1} & \frac{c}{c+1} \\
    0 & 0 & -1 & 0
\end{bmatrix}$\\

\vspace{3em}
$M_2:$ Scale far clipping plane
\vspace{1em}
$\begin{bmatrix}
    \frac{1}{\tan(\frac{\theta_w}{2})far} & 0 & 0 & 0 \\
    0 & \frac{1}{\tan(\frac{\theta_h}{2})far} & 0 & 0 \\
    0 & 0 & \frac{1}{far} & 0 \\
    0 & 0 & 0 & 1
\end{bmatrix}$

\vspace{3em}
$M_3:$ Rotate world to camera
\vspace{1em}
$\begin{bmatrix}
    u_x & u_y & u_z & 0 \\
    v_x & v_y & v_z & 0 \\
    w_x & w_y & w_z & 0 \\
    0 & 0 & 0 & 1
\end{bmatrix}$

\vspace{3em}
$M_4:$ Translate camera to the origin
\vspace{1em}
$\begin{bmatrix}
    1 & 0 & 0 & -e_x \\
    0 & 1 & 0 & -e_y \\
    0 & 0 & 1 & -e_z \\
    0 & 0 & 0 & 1
\end{bmatrix}$


\end{framed}
\noindent For the assignment you need to perform rotation operations on the camera in its own virtual $uvw$ coordinate system, e.g. spinning the camera about its $v$-axis. Additionally, you will need to perform translation operations on the camera in world space.
\begin{framed}
\noindent {\bf [1/2 pt. each]} How (mathematically) will you translate the camera's eye point $E$:
\begin{enumerate}
\item One unit right? $E' = $\\
    $T_{x+1} \cdot P_{uvw}$\\
\item One unit down? $E' = $\\
    $T_{y-1} \cdot P_{uvw}$\\
\item One unit forward? $E' = $\\
    $T_{z+1} \cdot P_{uvw}$\\
\end{enumerate}
\end{framed}

\noindent You can either move in and out of the camera coordinate space to perform these transformations, or you can do arbitrary rotations in world space. Both answers are acceptable.

\begin{framed}
\noindent {\bf [1/2 pt. each]} How (mathematically) will you use the $u$, $v$, and $w$ vectors, in conjunction with a rotation angle $\theta$, to get new $u$, $v$, and $w$ vectors when:
\begin{enumerate}
\item Adjusting the ``spin'' in a clockwise direction by $\theta$ radians?\\
    $R_{w-\theta} \cdot P_{uvw}$\\
\item Adjusting (rotating) the ``pitch'' to face upwards by $\theta$ radians?\\
    $R_{u\theta} \cdot P_{uvw}$\\
\item Adjusting the ``yaw'' to face right by $\theta$ radians?\\
    $R_{v\theta} \cdot P_{uvw}$\\
\end{enumerate}
\end{framed}

\section{Inverting Transformations}
\begin{framed}
\noindent {\bf [1 point]} Computing a full matrix inverse isn't always the most efficient way of inverting a transformation if you know all the components. Write out fully (i.e., write out all the elements) a 4x4 translation matrix $M_T$ and its inverse, $M^{-1}_T$. Our claim should become obvious to you.
\end{framed}

\begin{framed}
    $M_T$ = $\begin{bmatrix}
        1 & 0 & 0 & dx \\
        0 & 1 & 0 & dy \\
        0 & 0 & 1 & dz \\
        0 & 0 & 0 & 1
    \end{bmatrix}$\\\\
    $M_T^{-1}$ = $\begin{bmatrix}
        1 & 0 & 0 & -dx \\
        0 & 1 & 0 & -dy \\
        0 & 0 & 1 & -dz \\
        0 & 0 & 0 & 1
    \end{bmatrix}$
\end{framed}


{\bf Extra credit:} Will this same technique work for scale and rotation matrices? If so, write out the inverse scale and inverse rotation matrices. If not, explain why not. What about the perspective ``unhinging'' transformation? Will this technique be as efficient? Explain.

\begin{framed}
    Yes, similar techniques will work for the scaling and rotation matrices. For an inverse scaling, you multiply by the inverse of the scaling factor. For the rotation matrix, the inverse rotation is the transpose of the original rotation matrix.\\

\end{framed}

\section{How to Submit}

Hand in a PDF version of your solutions using the following command:
\begin{center}
 {\tt provide comp175 a2-alg}
 \end{center}
\end{document}  
